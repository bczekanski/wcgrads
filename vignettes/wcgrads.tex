% !TeX root = RJwrapper.tex
\title{Williams College Graduates}
\author{by Ben Czekanski}

\maketitle

\abstract{%
This package cleans and examines data found in Williams College Course
Catalogs about graduating seniors from 2000-2016. The data in this
package includes the names, graduation years and predicted genders of
graduates, as well as information about latin and departmental honors.
In addition to the data \textbf{wcgrads} provides examples of
interesting analyses of the data as examples for users.
}

\section{Introduction}\label{introduction}

The package \textbf{wcgrads} examines data on graduating seniors from
2000 to 2016 published in the Williams College Course Catalog. The
Catalog is published as a PDF document, so this package converts catalog
from .pdf to .txt and then cleans it into a tidy format for analysis.
\textbf{wcgrads} includes analyses of different elements of this data.
These analyses cover graduates initials' and name lengths, as well as
the distribution of latin honors over time. While the included analyses
are interesting and informative, \textbf{wcgrads} allows the user to
perform their own examinations by including the cleaned version of these
catalogs.

\begin{center}\rule{0.5\linewidth}{\linethickness}\end{center}

\section{Data}\label{data}

The data for this package comes from Course Catalogs posted online by
the Williams College Registrar. These files are first downloaded and
then converted to pdf using the pdftotext executable. After being
converted to .txt, the catalog is cut down to just the list of
graduating seniors and then cleaned into a single line of data per
graduate. Once in this single line format, all graduating classes are
combined into a single data frame of all students that graduated from
2000-2016. This data frame is called ``allyrs'' and it contains the
Catalogs' information on name, Phi Beta Kappa, graduation year, latin
honors, and departmental major. In addition ``allyrs'' adds gender to
the information provided in the Catalogs. It is important to note that
this is gender predicted by the R package \textbf{gender} using the
method ``ssa'' and is almost certainly not completely correct. However,
providing predicted gender gives package users additional, interesting
information to analyze.

\begin{center}\rule{0.5\linewidth}{\linethickness}\end{center}

\section{Analyses}\label{analyses}

The first question we can answer using this data is ``What is the
longest name?''. This question can be answered using the namelength()
function.

\begin{Schunk}
\begin{Sinput}
library(wcgrads)
names <- namelength()
head(names, 10)
\end{Sinput}
\begin{Soutput}
#> # A tibble: 10 × 3
#>                                     entirename namelength grad.year
#>                                          <chr>      <int>     <dbl>
#> 1  Susannah Sarah Mahinaokapolanialoha Fyrberg         40      2000
#> 2    William Oliver Sweetwater Parker Bobseine         37      2010
#> 3      JenniferAnne Lorraine McLellan Morrison         36      2012
#> 4       Oloruntosin Adepeju Ifedadepo Adeyanju         35      2008
#> 5       Christophe Alexander Dorsey-Guillaumin         35      2010
#> 6      Pierre-Alexandre Charles Meloty-Kapella         35      2010
#> 7       Bhuvaneswari Ettinamane Narendra Reddy         35      2011
#> 8     Chloë Iambe Naomi Illyria Feldman Emison         35      2012
#> 9        Katharine Elizabeth Huntress Peterson         34      2008
#> 10       Quinn Christianna Sanborn Brueggemann         34      2010
\end{Soutput}
\end{Schunk}

Next we can answer the question ``How has the number of graduating
seniors changed over time?''. The gradcount() function can answer this
question easily by provinding a graph showing the change over time.

\begin{Schunk}
\begin{Sinput}
library(wcgrads)
gradcount()
\end{Sinput}

\includegraphics{wcgrads_files/figure-latex/unnamed-chunk-2-1} \end{Schunk}

Yet another question that the allyrs data allows us to answer is ``How
has the distribution of latin honors changed over time?''.
\textbf{wcgrads} has a function for that as well. latin\_honors() shows
the percentage of students that received each level of latin honors by
year. There is no general trend, but there is a decent amount of
variation. Perhaps the variation can be explained by equal GPA's and the
methodology used in the case of such ties.

\begin{Schunk}
\begin{Sinput}
library(wcgrads)
latin_honors()
\end{Sinput}

\includegraphics{wcgrads_files/figure-latex/unnamed-chunk-3-1} \end{Schunk}

Delving further into latin honors, we find that the following in the
Course Catalogs:

``The Faculty will recommend to the Trustees that the degree of Bachelor
of Arts with distinction be conferred upon those members of the
graduating class who have passed all Winter Study Projects and obtained
a four year average in the top: 35\% of the graduating class ---
Bachelor of Arts cum laude or higher 15\% of the graduating class ---
Bachelor of Arts magna cum laude or higher 2\% of the graduating class
--- Bachelor of Arts summa cum laude''

The total\_latin\_honors() function shows the percentage distributions
of cumulative latin honors, and compares them to the guidelines(in
black) found in the Course Catalogs. Again there is no real trend, but
there is inconsistency. For all of the levels of honors there are years
where the distribution exceeds the guidelines, and years where it fails
to reach the guideline. This is confusing because one would expect that
if a tie did not allow the exact percentage to earn a certain level,
there would be consistent rounding up or rounding down, such that the
cumulative percentages are consistently above or below that
recommendation. This variation causes concern because these honors are
distinctions that appear on resumes, and that people work hard to
achieve. It therefore seems unfair that latin honors are given liberally
in some years and conservatively in others, without a clear methodology.

\begin{Schunk}
\begin{Sinput}
library(wcgrads)
total_latin_honors()
\end{Sinput}

\includegraphics{wcgrads_files/figure-latex/unnamed-chunk-4-1} \end{Schunk}

A simple investigation into how Phi Beta Kappa has been awarded over
time reveal that between twelve and thirteen percent of students
graduate Phi Beta Kappa, and that level has stayed fairly constant, with
some variation.

\begin{Schunk}
\begin{Sinput}
library(wcgrads)
pbk()
\end{Sinput}

\includegraphics{wcgrads_files/figure-latex/unnamed-chunk-5-1} \end{Schunk}

Two of the remaining unused variables in the allyrs dataset are
departmental honors and predicted gender. The dept\_honors() function
gives a chart that shows the raw number of graduates with honors from
each department, and their gender skew. Again, it is important to note
that this is predicted gender, but the figure still likely gives an
approximation of the makeup of each department. Biology has the most
graduates with some level of honors, and it appears that Mathematics is
the most skewed by gender among those receiving honors.

\begin{Schunk}
\begin{Sinput}
library(wcgrads)
dept_honors()
\end{Sinput}

\includegraphics{wcgrads_files/figure-latex/unnamed-chunk-6-1} \end{Schunk}

In addition to the departmental honors and gender, \textbf{wcgrads}
provides data about latin honors and gender. We can see that over this
time period slightly more men earned Summa Cum Laude than women, while a
few more women than men acheived Magna Cum Laude. The differences for
Cum Laude and no honors is much more. Significantly more women than men
earned Cum Laude, while more men than women did not earn latin honors.
From this chart we can infer that the average woman has a higher GPA
than the average man. However, at the Magna Cum Laude and Summa Cum
Laude levels, there are only minor differences, which suggests the
variance of male GPAs is larger than females'.

\begin{Schunk}
\begin{Sinput}
library(wcgrads)
latin_honors_gender()
\end{Sinput}

\includegraphics{wcgrads_files/figure-latex/unnamed-chunk-7-1} \end{Schunk}

A final, somewhat more trivial analysis of the data involves the
distribution of initials. The initials() function performs this analysis
and shows both the joint and marginal distribution of initials. Looking
at the plot generated, we see several main trends. First, the
distribution of last initials appears to be much more even than the
distribution of first initials. This suggests that there may be some
letters that are more popular than others. When given the choice of
first initial, letters A-E, J-N, and R-T are given more than other
letters. This could also simply be because there are more options for
first names that start with these initials. The difference between first
and last initals appears to be greatest in F-H and W, which all occur as
last initials much more often than they do as first initials. Another
interesting aspect of the graphic is observing which first initials are
more or less popular given a last initial. For example, L and K occur as
last initials at a similar rate. However, the pairing CL occurs much
more frequently than the pairing CK, possibly due to avoidance of
alliteration. Inversely, while E and K occur about as frequently as
first names, EC occurs much more than KC. The plot offers the
opportunity to make many other observations, and I invite the reader to
explore them. Further advancements on this subject could include showing
change over time, further breakdown by name, or breaking this down by
sound rather than letter to further examine how sound affects name
choice.

\begin{Schunk}
\begin{Sinput}
library(wcgrads)
initials()
\end{Sinput}

\includegraphics{wcgrads_files/figure-latex/unnamed-chunk-8-1} \end{Schunk}

\begin{center}\rule{0.5\linewidth}{\linethickness}\end{center}

\section{Conclusion}\label{conclusion}

The \textbf{wcgrads} package uses the ``allyrs'' data to answer a number
of questions, both important and trivial about Williams College
graduates. The limited analyses included in the package also raise
interesting questions ranging from how names are chosen to how the
College handles ties in GPA when awarding latin honors. More important
than any question it asks or answers, \textbf{wcgrads} makes information
buried in the Williams College Course Catalog easily available to anyone
who wants to analyze it.

\address{%
Ben Czekanski\\
Middlebury College '18\\
\\
}
\href{mailto:bczekanski@middlebury.edu}{\nolinkurl{bczekanski@middlebury.edu}}

